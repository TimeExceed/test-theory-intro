\documentclass[lualatex]{beamer}
\usepackage[english]{babel}
\usepackage{graphicx}
\usepackage{minted}

\setbeameroption{show notes}
\usetheme{Madrid}
\setminted{fontsize=\footnotesize}%fontfamily=Inconsolata,

\title[Theoretical Test Theory]{A Theoretical Introduction to Test Theory,\\And Its Applications}
  \author{Taoda}
  \institute{Alibaba Cloud}
  \date{Sep.\ 2017}

\begin{document}

\begin{frame}
\titlepage
\end{frame}

\begin{frame}
  \frametitle{Outline}
  \tableofcontents
\end{frame}

\section{A sample}

\begin{frame}[fragile]
  \frametitle{A sample}

  \begin{minted}[gobble=4]{python}
    def test_pk_order_insensitive(self):
        """...some explanation about what to do"""
        table_name = 'XXE' + self.get_python_version()
        table_meta = TableMeta(table_name, [('PK1', 'STRING'), ('PK2', 'INTEGER')])
        reserved_throughput = ReservedThroughput(CapacityUnit(
            restriction.MinReadWriteCapacityUnit,
            restriction.MinReadWriteCapacityUnit
        ))
        table_options = TableOptions()
        self.client_test.create_table(table_meta, table_options, reserved_throughput)
        self.wait_for_partition_load(table_name)
 
        pks = [('PK2', 123), ('PK1', 'blah')]
        try:
            self._check_all_row_op_without_exception(table_name, pks, [('C1', 'V')])
            self.assertTrue(False)
        except:
            pass
  \end{minted}
\end{frame}

\begin{frame}[fragile]
  \frametitle{A sample (cont.)}

  \begin{minted}[gobble=4]{python}
    def _check_all_row_op_without_exception(self, table_name, pks, cols):
        cu, row,token = self.client_test.get_row(table_name, pks, max_version=1)
        self.assert_equal(row, None)
        
        row = Row(pks, cols)
        cu,return_row = self.client_test.put_row(table_name, row, Condition(RowExistenceExpectation.IGNORE))
        cu, return_row, token = self.client_test.get_row(table_name, pks, max_version=1)
        self.assert_equal(return_row.primary_key, pks)
        self.assert_columns(return_row.attribute_columns, cols)

        row = Row(pks, {'put':cols})
        cu,return_row = self.client_test.update_row(table_name, row, Condition(RowExistenceExpectation.IGNORE))
        cu, rrow,token = self.client_test.get_row(table_name, pks, max_version=1)
        self.assert_equal(rrow.primary_key, pks)
        self.assert_columns(rrow.attribute_columns, cols)

        # other 70 lines
  \end{minted}
\end{frame}

\section{The Theory of Test}

\begin{frame}
  \frametitle{What is testing?}

  \begin{block}{Wikipedia says}
    Software testing is an investigation conducted to provide stakeholders with \emph{information about the quality} of the software product or service under test.
  \end{block}
  \pause
  \begin{center}
    \Huge
    \textcolor{red}{This is wrong!}
  \end{center}
\end{frame}

\begin{frame}
  \frametitle{What is testing?}
  
  \begin{block}{I say}
    A test is an executable specification, which defines behaviour of a trial under a certain type of scenes.
    \begin{itemize}
    \item trial
    \item oracle
    \item stimulus
    \item testbench
    \end{itemize}
  \end{block}
\end{frame}

\begin{frame}
  \frametitle{Trial}

  \begin{block}{~}
    \begin{itemize}
    \item A unit is not a trial, even it is a unit-test.
    \item An API is a trial.
    \item So, it is incorrect for a PutRow case to test GetRow/BatchGetRow/filters/...
    \end{itemize}
  \end{block}
\end{frame}

\begin{frame}
  \frametitle{Oracle}

  \begin{block}{~}
    \begin{itemize}
    \item Oracle is a theoretical concept, which is a marvelling, one-purpose computation device, whenever some inputs are put on it, a result comes up in a flash.
    \item 3 flavours:
      \begin{enumerate}
      \item is-checker
      \item equal-checker
      \item verifier
      \end{enumerate}
    \end{itemize}
  \end{block}
\end{frame}

\begin{frame}[fragile]
  \frametitle{Testa}

  \url{https://github.com/TimeExceed/testa}

  \begin{minted}[gobble=4]{erlang}
    testa:is(
      fun(_) -> 1 + 2 end,
      3),
    testa:eq(
      fun(_, {X,Y}) -> X*Y end,
      fun(_, {X,Y}) -> lists:sum([X || _ <- lists:seq(1, Y)]) end,
      fun(_, X) -> multipleTb(X) end),
    testa:verify(
      fun(_, {X,Y}) -> gcd(X, Y) end,
      fun(_, Result, {X, Y}) ->
        if (X rem Result =:= 0) and (Y rem Result =:= 0) -> ok;
           true -> {error, io_lib:format("~p/=gcd(~p,~p)", [Result, X, Y])}
        end
      end,
      fun(_, X) -> gcdtb(X) end)
  \end{minted}
\end{frame}

\begin{frame}
  \frametitle{Testbench \& Stimulus}

  \begin{block}{~}
    \begin{itemize}
    \item In EE
      \begin{itemize}
      \item stimulus: a sequence of 0-1 on control and data pins
      \item testbench: a board that provides suitable power, clocks and stimulus
      \end{itemize}
    \item in software:
      \begin{itemize}
      \item stimulus: concrete scenes of a certain type
      \item testbench: combines everything into an executable test, and provides stimulus
      \end{itemize}
    \end{itemize}
  \end{block}
\end{frame}

\begin{frame}[fragile]
  \frametitle{Testa}

  \begin{minted}[gobble=4]{erlang}
    gcdtb_(_, []) -> ok;
    gcdtb_(Case, [H|T]) -> 
      case Case(H) of
        ok -> gcdtb_(Case, T);
        {error, Msg} -> {error, Msg}
      end.
    gcdtb(Case) ->
      Ins = [{X, Y} || X <- lists:seq(0, 12),
                       Y <- lists:seq(0, 12),
                       (X /= 0) or (Y /= 0)],
      gcdtb_(Case, Ins).
  \end{minted}
\end{frame}

\section{Sample Revisit}

\begin{frame}[fragile]
  \frametitle{A sample}

  \begin{minted}[gobble=4]{python}
    def test_pk_order_insensitive(self):
        """...some explanation about what to do"""
        table_name = 'XXE' + self.get_python_version()
        table_meta = TableMeta(table_name, [('PK1', 'STRING'), ('PK2', 'INTEGER')])
        reserved_throughput = ReservedThroughput(CapacityUnit(
            restriction.MinReadWriteCapacityUnit,
            restriction.MinReadWriteCapacityUnit
        ))
        table_options = TableOptions()
        self.client_test.create_table(table_meta, table_options, reserved_throughput)
        self.wait_for_partition_load(table_name)
 
        pks = [('PK2', 123), ('PK1', 'blah')]
        try:
            self._check_all_row_op_without_exception(table_name, pks, [('C1', 'V')])
            self.assertTrue(False)
        except:
            pass
  \end{minted}
\end{frame}

\begin{frame}[fragile]
  \frametitle{A sample (cont.)}

  \begin{minted}[gobble=4]{python}
    def _check_all_row_op_without_exception(self, table_name, pks, cols):
        cu, row,token = self.client_test.get_row(table_name, pks, max_version=1)
        self.assert_equal(row, None)
        
        row = Row(pks, cols)
        cu,return_row = self.client_test.put_row(table_name, row, Condition(RowExistenceExpectation.IGNORE))
        cu, return_row, token = self.client_test.get_row(table_name, pks, max_version=1)
        self.assert_equal(return_row.primary_key, pks)
        self.assert_columns(return_row.attribute_columns, cols)

        row = Row(pks, {'put':cols})
        cu,return_row = self.client_test.update_row(table_name, row, Condition(RowExistenceExpectation.IGNORE))
        cu, rrow,token = self.client_test.get_row(table_name, pks, max_version=1)
        self.assert_equal(rrow.primary_key, pks)
        self.assert_columns(rrow.attribute_columns, cols)

        # other 70 lines
  \end{minted}
\end{frame}

\begin{frame}
  \frametitle{Mistakes in the sample}

  \begin{block}{On the good side}
    \begin{itemize}
    \item meaningful case name
    \end{itemize}
  \end{block}
  \begin{block}{On the bad side}
    \begin{itemize}
    \item no specific trial
    \item scene is not clear
    \item mixes trial and oracle
    \item too long test function
    \item conclusion: check-change-check anti-pattern
    \end{itemize}
  \end{block}
\end{frame}

\section{Good stimulus and good oracle}

\subsection{QuickCheck}

\begin{frame}[fragile]
  \frametitle{QuickCheck}

  \url{http://www.cse.chalmers.se/~rjmh/QuickCheck/manual.html}

  \begin{block}{correct property}
    A simple example of a property definition is
    \begin{minted}[gobble=6]{haskell}
      prop_RevRev xs = reverse (reverse xs) == xs
        where types = xs::[Int]
    \end{minted}
    To check the property, we load this definition in to hugs and then invoke
    \begin{minted}[gobble=6]{console}
      Main> quickCheck prop_RevRev
      OK, passed 100 tests.
    \end{minted}
  \end{block}
\end{frame}

\begin{frame}[fragile]
  \frametitle{QuickCheck on incorrect properties}

  \begin{block}{incorrect property}
    When a property fails, QuickCheck displays a counter-example.
    For example, if we define
    \begin{minted}[gobble=4]{haskell}
      prop_RevId xs = reverse xs == xs
        where types = xs::[Int]
    \end{minted}
    then checking it results in
    \begin{minted}[gobble=4]{console}
      Main> quickCheck prop_RevId
      Falsifiable, after 1 tests:
      [-3,15]
    \end{minted}
  \end{block}
\end{frame}

\begin{frame}
  \frametitle{Good stimulus and good oracle}

  \begin{block}{~}
    \begin{itemize}
    \item Good oracle: a certain property, e.g.,
      \begin{itemize}
      \item for a storage: what to get is exactly what is written
      \item for a sort: result in increasing order
      \end{itemize}
    \item Good stimulus: verify that property efficiently
      \begin{itemize}
      \item randomized methods
      \item heuristics
      \end{itemize}
    \end{itemize}
  \end{block}
\end{frame}

\begin{frame}[fragile]
  \frametitle{How to generate a random string for testing}

  \begin{block}{the amateur way}
    \begin{minted}[gobble=6]{python}
      def random_str(max_length, alphabet):
        l = random.randint(0, max_length)
        return ''.join([random.choice(alphabet) for _ in range(l)])
    \end{minted}
  \end{block}
  \pause
  \begin{block}{my way}
    \begin{minted}[gobble=6]{python}
      def random_str(alphabet, terminator):
        return ''.join(list(
          itertools.takewhile(lambda x:x!=terminator,
            (random.choice(alphabet) for _ in itertools.repeat(None)))))
    \end{minted}
    Observation: for a long input which triggers a bug, there is usually a short one to do the same thing.
  \end{block}
\end{frame}

\subsection{Pex}

\begin{frame}
  \frametitle{Pex: a heuristic way to generate stimulus}

  \begin{block}{~}
    \url{http://research.microsoft.com/Pex}\\
    \url{http://www.pexforfun.com/}
  \end{block}
\end{frame}

\begin{frame}[fragile]
  \frametitle{Pex: a heuristic way to generate stimulus}

  \begin{columns}[t]
    \column{.45\textwidth}
    \begin{minted}[gobble=6]{c}
      void Complicated(int x, int y) {
        if (x == Obfuscate(y))
          error();
        else 
          return;
      }
    \end{minted}
    \column{.45\textwidth}
    \begin{minted}[gobble=6]{c}
      int Obfuscate(int y) {
        return (100+y)*567 % 2347;
      }
    \end{minted}
  \end{columns}
  
  \begin{block}{Dynamic symbolic execution, starting from Complicated, runs the code twice}
    \begin{enumerate}
    \item Call \mintinline{c}{Complicated()} with arbitrary values, e.g., x=-312, y=513
      \begin{itemize}
      \item Record branch condition ``\mintinline{c}{x != (100+y)*567%2347}''
      \item \mintinline{c}{error()} is not hit
      \item Compute values such that ``\mintinline{c}{x == (100+y)*567%2347}'' (using constraint solver)
      \end{itemize}
    \item Call \mintinline{c}{Complicated()} with computed values
      \begin{itemize}
      \item \mintinline{c}{error()} is hit
      \end{itemize}
    \end{enumerate}
  \end{block}
\end{frame}

\begin{frame}
  \frametitle{Pex: a heuristic way to generate stimulus}

  \begin{block}{~}
    Pex is essentially a coverage tool.
    But when it inlines the oracles, it can also find bugs.
  \end{block}
\end{frame}

\section{Delta Debugging}

\begin{frame}
  \frametitle{Debugging}

  \begin{center}
    \includegraphics[height=0.8\textheight]{why_program_fails.jpg}
  \end{center}
\end{frame}

\begin{frame}[fragile]
  \frametitle{A Mozilla case}

  \begin{minted}[gobble=4]{html}
    <td align=left valign=top>
    <SELECT NAME="op_sys" MULTIPLE SIZE=7>
    <OPTION VALUE="All">All<OPTION VALUE="Windows 3.1">Windows 3.1
    ...
  \end{minted}
  \begin{block}{~}
    In 1999, a Mozilla user reported a crash bug.
    The report is simple, reproducible and precise,
    except it contains a 896 lines of HTML.
    In a BugAThon, this case was reduced to a simple tag \mintinline{html}{<SELECT>}.
    \pause
    \begin{center}
      How did they do?
    \end{center}
  \end{block}
\end{frame}

\begin{frame}[t]
  \frametitle{Delta Debugging: an example}

  \newcommand{\I}{\textcolor{blue}{|}}
  \tt
  \only<1>{~1~~2~~\textcolor{red}{3}~~4~~5~~6~~7~~\textcolor{red}{8}~~9~~10~~11~~12~~13~~14~~15~~16}
  \only<2>{\I1~~2~~\textcolor{red}{3}~~4~~5~~6~~7~~\textcolor{red}{8}\I~9~~10~~11~~12~~13~~14~~15~~16}
  \only<3>{~1~~2~~\textcolor{red}{3}~~4~~5~~6~~7~~\textcolor{red}{8}~\I9~~10~~11~~12~~13~~14~~15~~16\I}
  \only<4->{~1~~2~~\textcolor{red}{3}~~4~~5~~6~~7~~\textcolor{red}{8}~~9~~10~~11~~12~~13~~14~~15~~16}

  \only<4>{~1~~2~~\textcolor{red}{3}~~4~~5~~6~~7~~\textcolor{red}{8}}
  \only<5>{\I1~~2~~\textcolor{red}{3}~~4\I~5~~6~~7~~\textcolor{red}{8}}
  \only<6>{~1~~2~~\textcolor{red}{3}~~4~\I5~~6~~7~~\textcolor{red}{8}\I}
  \only<7>{\I1~~2\I~\textcolor{red}{3}~~4~~5~~6~~7~~\textcolor{red}{8}}
  \only<8->{~1~~2~~\textcolor{red}{3}~~4~~5~~6~~7~~\textcolor{red}{8}}

  \only<8>{~\textcolor{red}{3}~~4~~5~~6~~7~~\textcolor{red}{8}}
  \only<9>{\I\textcolor{red}{3}~~4~~5\I~6~~7~~\textcolor{red}{8}}
  \only<10>{~\textcolor{red}{3}~~4~~5~\I6~~7~~\textcolor{red}{8}\I}
  \only<11>{\I\textcolor{red}{3}\I~4~~5~~6~~7~~\textcolor{red}{8}}
  \only<12>{~\textcolor{red}{3}~\I4\I~5~~6~~7~~\textcolor{red}{8}}
  \only<13->{~\textcolor{red}{3}~~4~~5~~6~~7~~\textcolor{red}{8}}

  \only<13>{~\textcolor{red}{3}~~5~~6~~7~~\textcolor{red}{8}}
  \only<14>{\I\textcolor{red}{3}~~5\I~6~~7~~\textcolor{red}{8}}
  \only<15>{~\textcolor{red}{3}~~5~\I6~~7\I~\textcolor{red}{8}}
  \only<16->{~\textcolor{red}{3}~~5~~6~~7~~\textcolor{red}{8}}

  \only<16>{~\textcolor{red}{3}~~5~~\textcolor{red}{8}}
  \only<17>{\I\textcolor{red}{3}\I~5~~\textcolor{red}{8}}
  \only<18>{~\textcolor{red}{3}~\I5\I~\textcolor{red}{8}}
  \only<19->{~\textcolor{red}{3}~~5~~\textcolor{red}{8}}

  \only<19>{~\textcolor{red}{3}~~\textcolor{red}{8}}
  \only<20>{\I\textcolor{red}{3}\I~\textcolor{red}{8}}
  \only<21>{~\textcolor{red}{3}~\I\textcolor{red}{8}\I}
  \only<22->{~\textcolor{red}{3}~~\textcolor{red}{8}}
\end{frame}

\begin{frame}
  \frametitle{Delta Debugging: some experiments}

  \begin{block}{Still long after DD shrinking?}
    \begin{itemize}
    \item cache
    \item nondeterministics (.e.g., randomized or multi-threaded)
    \end{itemize}
  \end{block}
\end{frame}

\section{Conclusion}

\begin{frame}
  \frametitle{Conclusion}

  \begin{block}{We present}
    \begin{itemize}
    \item The theory of test
      \begin{itemize}
      \item trial
      \item oracle
      \item testbench
      \item stimulus
      \end{itemize}
      \pause
    \item How this theory improve readability
      \begin{itemize}
      \item Testa
      \item QuickCheck
      \end{itemize}
      \pause
    \item How this theory improve quality
      \begin{itemize}
      \item randomized stimulus generator, binomial distribution
      \item heuristic stimulus generator, Pex
      \end{itemize}
      \pause
    \item How this theory make debugging efficiently
      \begin{itemize}
      \item delta debugging
      \end{itemize}
    \end{itemize}
  \end{block}
\end{frame}

\section{Q\&A}

\begin{frame}
  \frametitle{}
  \begin{center}
    \Huge
    Questions?
  \end{center}
\end{frame}

\end{document}
